\section{Termine}\label{sec:schedule}

\begin{table}

\caption{Zeitplan}\label{tab:schedule}

\begin{tabularx}{\textwidth}{lX}
Datum & Ziele \\
\hline
01.05.2019 & Technologien bestimmt\\
08.05.2019 & Technologien installiert und lauffähig; Zeit- und Architekturplan erstellt\\
15.05.2019 & Datenmodell ausgearbeitet; GUI auf Papier designed\\
22.05.2019 & \href{https://guides.rubyonrails.org/security.html}{Rails Sicherheitsguide} und \href{https://github.com/OWASP/CheatSheetSeries}{OWASP} überflogen; obligatorischen Teil des Datenmodells umgesetzt; Verbindung zwischen Front- und Backend hergestellt \\
29.05.2019 & Benutzerregistrierung ermöglicht; Prototypen erstellt; Vortragsfolien erstellt; Konzept für automatische Kontrolle erarbeitet\\
\hline
05.06.2019 & Dozentenview; Aufgabenstellungen erstellbar; automatische Aufgabenkontrolle getestet\\
12.06.2019 & Studentenview; Aufgaben lösbar; automatische Aufgabenkontrolle implementiert\\
19.06.2019 & Einstellungsmenü; freies Üben ermöglicht; Tests \\
26.06.2019 & Optionales (wie Leaderboard, Achievements) implementiert \\
\hline
03.07.2019 & Puffer; Kernprojekt fertig; kleine Verbesserungen (z. B. Dokumentation)\\
10.07.2019 & Abschlussvortragsfolien erstellt\\
\end{tabularx}

\end{table}
