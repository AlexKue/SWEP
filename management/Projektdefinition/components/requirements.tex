\section{Anforderungen}\label{sec:requirements}

\subsection{Funktionale Anforderungen}

\begin{itemize}
  \item Kontenverwaltung von Studierenden und Dozierenden
  \item Dozierende können Aufgaben erstellen
  \item Studierende können Aufgaben bearbeiten
  \begin{itemize}
    \item Aufgabenstellung in natürlicher Sprache
    \item Entgegennahme von SQL-Anfragen
    \item Anzeige des Ergebnisses der Anfrage (auf zufälliger/ Beispieltabelle)
    \item Erkennung der Korrektheit der Anfrage bezüglich der gestellten Frage
    \item bei Unsicherheit $\rightarrow$ Ersteller informieren
  \end{itemize}
  \item freies Üben
  \item Studierende sollen virtuelle Abzeichen erwerben können $\rightarrow$ von Dozenten erstellt
  \item Adminaccount zur Benutzerverwaltung? $\rightarrow$ Nein; wird von Dozenten miterledigt
\end{itemize}

\subsection{Nichtfunktionale Anforderungen}
\begin{description}
  \item [Einfachheit] Das Programm soll durch Benutzer ohne das Lesen einer Anleitung bedienbar sein.
  \item [Wartbarkeit] Das Programm soll auch für projektexterne Entwickler verständlich, wartbar und erweiterbar sein.
  \item [Qualität] Das Programm soll durch Tests grundlegenden Qualitätsansprüchen genügen.
  \item [Freiheit] Das Programm soll unter Linux funktionieren und möglichst nur auf freie/offene Software zurückgreifen.
\end{description}

