\chapter{Protokolle}

\section*{01. Mai 2019}
\paragraph{Ziele}
\begin{itemize}
  \item[\check] zu verwendende Technologien bestimmt
  \item[\check] grundlegenden Zeitplan entworfen
\end{itemize}

\section*{08. Mai 2019}
\paragraph{Ziele}
\begin{itemize}
  \item[\check] Programmbibliotheken installiert und eingerichtet
  \item[\check] Liefergegenstände spezifiert
  \item[\check] Zeitplan daran angepasst
  \item[\check] Zuständigkeiten geklärt
\end{itemize}

\paragraph{Verlauf}
\begin{itemize}
  \item Dozenzen = Admins
  \item Spielwiese approved
  \item Timeout bei Anfrage setzen
  \item nach Einloggen Aufgabenliste (möglicherweise nach Kategorien geordnet) anzeigen $\rightarrow$ mit Häkchen dran
  \item Leaderboard
  \item Programm als 1-Page-Anwendung
\end{itemize}

\section*{15. Mai 2019}
\paragraph{Ziele}
\begin{itemize}
  \item[\check] Datenmodell ausgearbeitet
  \item[\check] GUI-Skizze auf Papier erstellt
\end{itemize}
\paragraph{Verlauf}
\begin{itemize}
  \item Menü oben oder an der Seite des GUIs
  \item Speichern der letzten Query pro Aufgabe, aber nur explizit mit einem Knopf
  \item Kein \glqq{}Sind Sie sich sicher, dass Sie die Seite ohne Speichern verlassen wollen?\grqq{}-Pop-up
  \item Ergebnis von eingereichten Lösungen $\in \{\texttt{richtig}, \texttt{falsch}, \texttt{unsicher}\}$
  \item Einsehen der Ergebnisse durch Betreuer, der bei \texttt{unsicher} selbst entscheiden kann
  \item Hinterlegen der DB-Schemen TPC-H und Uni mit Daten; Ausführen der Anfragen darauf
  \item Hinterlegen eines \glqq{}versteckten\grqq{} Datensatzes zur zusätzlichen Validierung der Querys
  \item Abstufen der Priorität von Scoreboard, Abzeichen und Rücksetzen des Passworts
  \item Speichern von mehreren Musterquerys pro Aufgabe als Dozent
  \item Überprüfen der Querys z. B. mit Postgres \texttt{EXPLAIN}
  \item Berechnen der Punkte eines Benutzers (nicht speichern) für bessere Konsistenz
  \item Rechtemanagement auf Anwendungsebene statt DB-Ebene
\end{itemize}

\section*{22. Mai 2019}
\paragraph{Ziele}
\begin{itemize}
  \item Prototypen erstellt
  \begin{itemize}
    \item Aufgabe auswählen
    \item Query eintragen
    \item Ergebnis wird angezeigt zusammen mit "Richtig!"
  \end{itemize}
  \item Vortrag ausgearbeitet
  \begin{itemize}
    \item Aufgabenstellung
    \item Vorgehen/Vision
    \item Stand
    \item Zukunftspläne
    \item Vortragsfolien erstellt
  \end{itemize}
\end{itemize}
